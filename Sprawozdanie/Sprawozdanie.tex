\documentclass[a4paper,titleauthor]{mwart} 
\usepackage{polski}
\usepackage[utf8]{inputenc}
\usepackage{graphicx} %pakiet do wstawiania grafiki
\usepackage[hyphens]{url} %pakiet do wstawiania linkow
%\usepackage[hidelinks,breaklinks]{hyperref}
\usepackage{authblk}%pakiet do tworzenia afiliacji
\usepackage{tabularx}%pakiet do tabel
\usepackage[a4paper, left=2cm, right=2cm, top=3cm, bottom=3cm]{geometry}
\usepackage{listings}
\usepackage{placeins}%pakiet do kontroli umieszczania obiektow
\usepackage{hyperref}%pakiet do m.in. kolorowania linkow
\usepackage{fancyhdr}
\usepackage{float} 
\usepackage{hyperref}
\usepackage[tablegrid,owncaptions]{vhistory}
\usepackage{subfigure}
\usepackage{listings}
\usepackage{wrapfig}
\usepackage{supertabular}
\usepackage{}
\usepackage[polish]{babel}
\usepackage{amsmath,amssymb}
\usepackage{caption}
\usepackage[T1]{fontenc}
\usepackage{multicol}
\usepackage{booktabs}
\renewcommand\figurename{Rys.}%skrocony podpis
\renewcommand\lstlistingname{Wydruk}


%------------------------------------------------------------------------
% Dane do strony tytułowej
% Początek dokumentu
\begin{document}
	\begin{titlepage}
		\newgeometry{top = 0.5in,right = 0.5in, left=0.5in, bottom=1in}
		
		\begin{center}
			Politechnika Warszawska \\
			Wydział Geodezji i Kartografii
		\end{center}
		
		\hrule
		\vspace*{1cm}
		\begin{center}
			\Large{\textbf{Informatyka geodezyjna II}}
		\end{center}
		
		
		\vspace*{2cm}
		\begin{center}
			\large{\textbf{Projekt 2}} 
		\end{center}
		\vspace{3cm}
		\hrule
		
		\begin{center}
			\Large{\textbf{Wtyczka do QGIS - PyQGIS}}
		\end{center}
		\hrule
		
		\vspace*{2cm}
		\begin{center}
			\large{Michał Bielecki \ \ \ \ 319294}\\ 
			\large{Michał Chwałek \ \ \ \ 319305} 
		\end{center}
		
		\vspace*{3cm}
		
		\begin{center}
			\normalsize{\textbf{Grupa 1}}\\
			\small{Zajęcia: \\poniedziałek 12:15-14:00} \\
			\small{Rok akademicki:\\ 2022/23, Semestr 4}
		\end{center}
		
		\vspace*{3cm}
		\hrule
		\begin{center}
			\large{\textbf{Prowądzacy:} \ mgr inż. Andrzej Szeszko}
		\end{center}
		\hrule
		
		
	\end{titlepage}
	
	%---------------------------------------------------------------
\newpage
%Automatycznie generowany spis treści
\tableofcontents
\newpage
%------------------------------------------------
	
\section{Cel ćwiczenia} 
	Celem ćwiczenia jest napisanie wtyczki do programu QGiS umożliwiającej obliczenie różnicy wysokości pomiędzy dwoma punktami wybranymi z warstwy lub obliczenia powierzchni wieloboku utworzonego z conajmniej 3 zaznaczonych punktów. 
\section{Wykorzystane oprogramowanie oraz moduły}
Do utworzenia UserInterface naszej wtyczki został wykorzystany program \textbf{QT Designer with QGIS custom widgets}, funkcjonalność wtyczki została napisana z wykorzystaniem języka programowania \textbf{python} w wersji \textbf{3.9.7}. Do opracowania wtyczki użyliśmy modułów:
\begin{itemize}
	\item \textbf{qgis.PyQt}
	\item \textbf{qgis.core}
	\item \textbf{qgis.utils}
\end{itemize}
W celu konwersji pliku interfejsu o rozszerzeniu \textbf{.ui} do pliku o rozszerzeniu \textbf{.py} użyliśmy narzędzia \textbf{pyuic5}. W celu utworzenia oraz testowania wtyczki użyliśmy pluginów do QGiS:
\begin{itemize}
	\item \textbf{Plugin Builder}
	\item \textbf{Plugin Reloader}
\end{itemize}
Do testowania poprawności działania wtyczki użyliśmy stworzonej przez nas warstwy testowej, w której atrybuty geometrii (współrzędne X,Y,Z) znajdowały się w tabeli atrybutów.
\section{Przebieg ćwiczenia}
Projekt rozpoczeliśmy od utworzenia wtyczki przy pomocy zainstalowanego pluginu do QGiS - \textbf{Plugin Builder}. Dzięki temu plugin utworzył nam pliki potrzebne do rozpoczęcia pracy nad User Interface oraz samym działaniem naszej wtyczki. Następnie w programie \textbf{QT Designer} rozpoczeliśmy prace nad wyglądem interface'u wtyczki. Dodaliśmy tam elementy, takie jak:
\begin{itemize}
	\item pushButton
	\item comboBox
	\item MapLayerComboBox
	\item radioButton
	\item label
\end{itemize}		
Po dodaniu elementóW oraz ich wstępnym rozmieszczeniu, przekonwertowaliśmy plik \textbf{.ui} do pliku \textbf{.py}. Następnie rozpoczeliśmy pracę nad funkcjonalnością wtyczki. Pierwszym krokiem było utworzenie modułu, który pozwalał na obliczenie różnicy wysokości między dwoma punktami wybranymi z warstwy. W tym celu wykorzystaliśmy punkty z naszej warstwy testowej. W celu poprawnego działania modułu na samym początku dodaliśmy warunek 
\begin{lstlisting}
	if self.comboBox_obliczenie.currentText() != 'Różnica wysokości':
		return
\end{lstlisting}
Dzięki temu wtyczka oblicza wysokość tylko w przypadku, gdy w comboBox'ie przeznaczonym do określenia typu obliczenia została wybrana opcja \textbf{\textit{Różnica wysokości}}.
\end{document}