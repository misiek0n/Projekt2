\documentclass[a4paper,titleauthor]{mwart} 
\usepackage{polski}
\usepackage[utf8]{inputenc}
\usepackage{graphicx} %pakiet do wstawiania grafiki
\usepackage[hyphens]{url} %pakiet do wstawiania linkow
%\usepackage[hidelinks,breaklinks]{hyperref}
\usepackage{authblk}%pakiet do tworzenia afiliacji
\usepackage{tabularx}%pakiet do tabel
\usepackage[a4paper, left=2cm, right=2cm, top=3cm, bottom=3cm]{geometry}
\usepackage{listings}
\usepackage{placeins}%pakiet do kontroli umieszczania obiektow
\usepackage{hyperref}%pakiet do m.in. kolorowania linkow
\usepackage{fancyhdr}
\usepackage{float} 
\usepackage{hyperref}
\usepackage[tablegrid,owncaptions]{vhistory}
\usepackage{subfigure}
\usepackage{listings}
\usepackage{wrapfig}
\usepackage{supertabular}
\usepackage{}
\usepackage[polish]{babel}
\usepackage{amsmath,amssymb}
\usepackage{caption}
\usepackage[T1]{fontenc}
\usepackage{multicol}
\usepackage{booktabs}
\renewcommand\figurename{Rys.}%skrocony podpis
\renewcommand\lstlistingname{Wydruk}


%------------------------------------------------------------------------
% Dane do strony tytułowej
% Początek dokumentu
\begin{document}
	\begin{titlepage}
		\newgeometry{top = 0.5in,right = 0.5in, left=0.5in, bottom=1in}
		
		\begin{center}
			Politechnika Warszawska \\
			Wydział Geodezji i Kartografii
		\end{center}
		
		\hrule
		\vspace*{1cm}
		\begin{center}
			\Large{\textbf{Informatyka geodezyjna II}}
		\end{center}
		
		
		\vspace*{2cm}
		\begin{center}
			\large{\textbf{Projekt 2}} 
		\end{center}
		\vspace{3cm}
		\hrule
		
		\begin{center}
			\Large{\textbf{Wtyczka do QGIS - PyQGIS}}
		\end{center}
		\hrule
		
		\vspace*{2cm}
		\begin{center}
			\large{Michał Bielecki \ \ \ \ 319294}\\ 
			\large{Michał Chwałek \ \ \ \ 319305} 
		\end{center}
		
		\vspace*{3cm}
		
		\begin{center}
			\normalsize{\textbf{Grupa 1}}\\
			\small{Zajęcia: \\poniedziałek 12:15-14:00} \\
			\small{Rok akademicki:\\ 2022/23, Semestr 4}
		\end{center}
		
		\vspace*{3cm}
		\hrule
		\begin{center}
			\large{\textbf{Prowądzacy:} \ mgr inż. Andrzej Szeszko}
		\end{center}
		\hrule
		
		
	\end{titlepage}
	
	%---------------------------------------------------------------
\newpage
%Automatycznie generowany spis treści
\tableofcontents
\newpage
%------------------------------------------------
	
\section{Cel ćwiczenia} 
	Celem ćwiczenia jest napisanie wtyczki do programu QGiS umożliwiającej obliczenie różnicy wysokości pomiędzy dwoma punktami wybranymi z warstwy lub obliczenia powierzchni wieloboku utworzonego z conajmniej 3 zaznaczonych punktów. 
\section{Wykorzystane oprogramowanie oraz moduły}
Do utworzenia UserInterface naszej wtyczki został wykorzystany program \textbf{QT Designer with QGIS custom widgets}, funkcjonalność wtyczki została napisana z wykorzystaniem języka programowania \textbf{python} w wersji \textbf{3.9.7}. Do opracowania wtyczki użyliśmy modułów:
\begin{itemize}
	\item \textbf{qgis.PyQt}
	\item \textbf{qgis.core}
	\item \textbf{qgis.utils}
\end{itemize}
W celu konwersji pliku interfejsu o rozszerzeniu \textbf{.ui} do pliku o rozszerzeniu \textbf{.py} użyliśmy narzędzia \textbf{pyuic5}. W celu utworzenia oraz testowania wtyczki użyliśmy pluginów do QGiS:
\begin{itemize}
	\item \textbf{Plugin Builder}
	\item \textbf{Plugin Reloader}
\end{itemize}
Do testowania poprawności działania wtyczki użyliśmy stworzonej przez nas warstwy testowej, w której atrybuty geometrii (współrzędne X,Y,Z) znajdowały się w tabeli atrybutów.
\section{Przebieg ćwiczenia}
Projekt rozpoczeliśmy od utworzenia wtyczki przy pomocy zainstalowanego pluginu do QGiS - \textbf{Plugin Builder}. Dzięki temu plugin utworzył nam pliki potrzebne do rozpoczęcia pracy nad User Interface oraz samym działaniem naszej wtyczki. Następnie w programie \textbf{QT Designer} rozpoczeliśmy prace nad wyglądem interface'u wtyczki. Dodaliśmy tam elementy, takie jak:
\begin{itemize}
	\item pushButton
	\item comboBox
	\item MapLayerComboBox
	\item radioButton
	\item label
\end{itemize}		
Po dodaniu elementóW oraz ich wstępnym rozmieszczeniu, przekonwertowaliśmy plik \textbf{.ui} do pliku \textbf{.py}. Następnie rozpoczeliśmy pracę nad funkcjonalnością wtyczki. Pierwszym krokiem było utworzenie modułu, który pozwalał na obliczenie różnicy wysokości między dwoma punktami wybranymi z warstwy. W tym celu wykorzystaliśmy punkty z naszej warstwy testowej. W celu poprawnego działania modułu na samym początku dodaliśmy warunek 
\begin{verbatim}
	if self.comboBox_obliczenie.currentText() != 'Różnica wysokości':
		return
\end{verbatim}
Dzięki temu wtyczka oblicza wysokość tylko w przypadku, gdy w comboBox'ie przeznaczonym do określenia typu obliczenia została wybrana opcja \textbf{\textit{Różnica wysokości}}.
Analogiczny warunek powstał przy opcji wybierania obliczania pól powierzchni

W następnym kroku przypisaliśmy wybraną przez użytkownika warstwę do zmiennej \textbf{layer} i za pomocą metody \textbf{selectedFeatures()} pobieramy zaznaczone przez użytkownika punkty

\begin{verbatim}
	layer = self.mMapLayerComboBox_layers.currentLayer()
	selected_points = layer.selectedFeatures()
\end{verbatim}
Następnie za pomocą iteracji "wydobywamy" wartości z tabeli atrybutów z konkretnych punktów, w tym celu wykorzystujemy warunek zgodności nazwy.

\begin{verbatim}
	ID = []
	
	for point in selected_points:
	idnumber = point['ID']
	ID.append(idnumber)
\end{verbatim}

Z wydobytych wartości punktów napisaliśmy kod w celu obliczenia różnicy wysokości i powierzchni pola:\\ \\
$\bullet$ W celu obliczenia powierzchni pola algorytm najpierw sortuje punkty po współrzędnych w kierunku przeciwnym do ruchu wskazówek zegara, a następnie przy pomocy wzorów Gaussa liczy pole powierzchni\\
$\bullet$ W celu obliczenia różnic wysokości wykorzystaliśmy wzór: $h_{2} - h_{1}$

Zaznaczone przez użytkonika punkty wyświetlają się w czasie rzeczywistym w tabeli. W tym celu stworzyliśmy 3 metody: 
\begin{verbatim}
	update_table, update_table_for_surface_area, update_table_for_height_difference
\end{verbatim}
Metoda \textbf{UpdateTable} definiuje ogólny widok tabeli w zależności od wybranej opcji obiczeń, następnie odwołuje się do kolejnej metody w celu pobrania konkretnych wartości punktów.

\begin{verbatim}
	def update_table(self):
	selected_option = self.comboBox_obliczenie.currentText()
	self.tableWidget_wybrane.clearContents()
	self.tableWidget_wybrane.setRowCount(0)
	
	if selected_option == 'Pole powierzchni':
	self.tableWidget_wybrane.setColumnCount(3)
	self.tableWidget_wybrane.setHorizontalHeaderLabels(['Nr', 'X', 'Y'])
	self.update_table_for_surface_area()
	else:
	self.tableWidget_wybrane.setColumnCount(2)
	self.tableWidget_wybrane.setHorizontalHeaderLabels(['Nr', 'h'])
	self.update_table_for_height_difference()
\end{verbatim}

Metody \textbf{UpdateTableForSurfaceArea} oraz\textbf{UpdateTableForHeigtDifferences} iterują po posczególnych elemntach i wybierają pożądane wartości, a następnie przypisują je do odpowiednich kolumn w tabeli:

\begin{verbatim}
	self.tableWidget_wybrane.setRowCount(len(ID))
	
	for row, data in enumerate(zip(ID, Xcoord, Ycoord)):
	for col, value in enumerate(data):
	item = QtWidgets.QTableWidgetItem(str(value))
	self.tableWidget_wybrane.setItem(row, col, item)
\end{verbatim}
Metoda \textbf{CompareArea} iteruje po tabeli atrybutów w celu wydobycia pola powierzchni z wbudowanych w QGIS'ie narzędzi geometrii. Użytkownik następnie ma możliwość wybór jednostki wyświetlanej wartości:

\begin{verbatim}
	 def compare_area(self):
	layer = self.mMapLayerComboBox_layers.currentLayer()
	selected_feature = layer.selectedFeatures()
	layer.startEditing()
	for i in selected_feature:
	area = i.geometry().area()
	layer.changeAttributeValue(i.id(), layer.fields().indexFromName('Powierzchnia'), area)
	layer.commitChanges()
	if self.radioButton_m2.isChecked():
	self.label_porow.setText(f"Wynik to: {area} metrów kwadratowych")
	elif self.radioButton_ar.isChecked():
	self.label_porow.setText(f"Wynik to: {area / 100} arów")
	elif self.radioButton_ha.isChecked():
	self.label_porow.setText(f"Wynik to: {area / 10000} hektarów")
\end{verbatim}

Metoda \textbf{create\_polygon} sortuje wybrane przez użytkownika punkty zgodnie z ruchem wskazówek zegara, następnie pobiera parametry warstwy na której znajdują się punkty, a następnie tworzy nową warstwę na podstawie wcześniej pobranych parametrów, na której znajduje się poligon utworzony z wcześniej posortowanych punktów.

\begin{verbatim}
    def create_polygon(self):
		canvas = iface.mapCanvas()
		layer = self.mMapLayerComboBox_layers.currentLayer()
		crs = layer.crs()
		crs_authid = crs.authid()
		selected_features = layer.selectedFeatures()
		if len(selected_features) < 3:
		iface.messageBar().pushMessage("Błąd",
		"Wybierz co najmniej 3 punkty do narysowania poligonu.",
		level=Qgis.Warning)
		return
		points = [feature.geometry().asPoint() for feature in selected_features]
		centroid = QgsPoint(sum(point.x() for point in points) / len(points),
		sum(point.y() for point in points) / len(points))
		points.sort(key=lambda point: -math.atan2(point.y() - centroid.y(), point.x() - centroid.x()))
		new_layer = QgsVectorLayer("Polygon?crs=" + crs_authid, "Poligon", "memory")
		provider = new_layer.dataProvider()
		new_layer.startEditing()
		poly_feature = QgsFeature()
		polygon = QgsGeometry.fromPolygonXY([points])
		poly_feature.setGeometry(polygon)
		provider.addFeature(poly_feature)
		new_layer.commitChanges()
		QgsProject.instance().addMapLayer(new_layer)
canvas.refresh()
\end{verbatim}

\section{Wnioski}
Dzięki pracy przy projekcie mieliśmy możliwość rozwinąć swoje umiejętności w pracy z GitHub'em, QGiS'em, QT Designerem oraz pythonem. Niestety, przez niewystarczającą ilość czasu oraz nieubłagalnie zbliżającą się sesję nie daliśmy rady wykonać wszystkich dodatkowych opcji. Poniżej znajduje się link do zdalnego repozytorium.
\vspace{4cm}
\begin{center}
	\href{https://github.com/misiek0n/Projekt2}{github.com/misiek0n/Projekt2}
\end{center}


\end{document}

